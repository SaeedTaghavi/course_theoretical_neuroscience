%\documentclass[a4paper,12pt]{report}
\documentclass[a4paper,12pt]{article}


\usepackage{graphicx}
\usepackage[top=3cm,right=2.5cm,bottom=3cm,left=2.5cm]{geometry}  
\usepackage{fancyhdr}
\usepackage[colorlinks,linkcolor=blue,citecolor=blue,pagebackref=false]{hyperref}
\usepackage{amsmath}
\usepackage{graphicx}
\usepackage{hvfloat, subfig}
\usepackage{xepersian}
\setcounter{section}{0}

\settextfont[Scale=1]{XB Niloofar}
%\settextfont[Scale=1]{Nazli}
\setlatintextfont[Scale=1]{Times New Roman}
\setdigitfont[Scale=1]{XB Niloofar}
%\setdigitfont[Scale=1]{Nazli}

\lhead{\thepage}
\chead{}

\renewcommand{\headrulewidth}{0.4pt}
\renewcommand{\footrulewidth}{0.4pt}

%\fancypagestyle{fpstyle}{%
\fancyhead{} % get rid of headers
\renewcommand{\headrulewidth}{0pt} % and the line

\begin{document}
\begin{center}
{\Huge مبانی نظری علوم اعصاب 
\\
\begin{latin}
overview \& Neurons I
\end{latin}
}
\vspace*{1cm}
سعید تقوی
 \LTRfootnote{Email: s.taghavi@iasbs.ac.ir}

% دانشكده فیزیك، دانشگاه تحصیلات تكمیلی علوم پایه زنجان، صندوق پستی ۱۱۵۹-۴۵۱۹۵، زنجان، ایران\\

% {\small
% \begin{minipage}{12cm}
% \vspace*{1cm}
% متن چکیده
% \end{minipage}
% }
\end{center}
\section*{ }
مهم‌ترین پرسش علوم شناختی محاسباتی، چگونگی پردازش اطلاعات در سطوح مختلف مغز است. موضوعات و مسائل اصلی مورد مطالعه و پژوهش در حوزه پهناور علوم اعصاب محاسباتی را می‌توان به صورت‌های گوناگون شمارش و دسته‌بندی کرد. در این دوره از بین حوزه های متفاوت علوم اعصاب محاسباتی، تمرکز ما بیشتر بر روی مدل سازی تک نورون، مدل سازی جمعیت های نورونی و فعالیت شبکه، حافظه و انعطاف پذیری سیناپسی و بررسی مسیر اطلاعات در شبکه های نورونی می باشد.
فرآیند های زیر سلولی نظیر تولید و انتقال ناقل های عصبی، شکل و هندسه سلول عصبی، نحوه کارکرد کانال های یونی و موارد دیگر بسیار پیچیده هستند، با این حال ما می توانیم بررسی خود را از مدل سازی تک نورون آغاز کنیم.  
\\
سلول های عصبی را دو دسته سلول که از دید ساختاری کاملن ناهمسان هستند تشکیل می دهند؛ این دو دسته عبارتند از:
\begin{itemize}
	\item{
	 سلول های تحریک پذیر (نورون ها) که مسئول پردازش و انتقال اطلاعات هستند.
	}
	\item {
	 سلول های تحریک ناپذیر (سلول های گِلیال در بر گیرنده:آستروسیت‌ها، میکروگلیال‌ها، و الیگودندروسیت‌ها) و سلول شوان نام برد؛ که وظیفه حمایت و تغذیه دستگاه عصبی را بر عهده دارند.
	}
\end{itemize}
نورون ها سه بخش اصلی دارند: جسم سلولی، دندریت و آکسون. جسم سلولی بخشی از نورون است که دارای هسته بوده و تولید موادی را که نورون برای رشد و انجام وظایفش به آنها نیااز دارد تنظیم می کند. دندریت، انشعابات سلول عصبی که اطلاعات را دریافت می کند و آن را به جسم سلولی می برد. آکسون، بخشی از سلول عصبی است که اطلاعات را از جسم سلولی به سمت سلول های دیگر هدایت می کند. در اغلب اوقات هر سلول عصبی فقط یک آکسون دارد. سه دسته کلی نورون داریم: نورون های حسی (آوران)، حرکتی (وابران) و میانجی (ارتباطی). نورون های آوران، تحریکات حسی دریافت شده توسط گیرنده های حسی را به دستگاه عصبی می رسانند. نورون های وابران، حامل پیام هایی هستند که از مغز یا نخاع به اعضای پاسخ دهنده مانند عضلات و یا غده ها می روند. نورون های ارتباطی، از اندام های حسی پیام هایی دریافت می کنند و تکانه هایی را به سایر نورون های میاجی یا نورون های حرکتی می فرستند. این نورون ها فقط در مغز و چشم و نخاع وجود دارد.
نورون ها تقسیم نمی‌شوند.

نورون‌ها، اصلی‌ترین سلول های عصبی هستند. بین سلول های بدن، نورون ها به خاطر قابلیت شان در انتشار سریع سیگنال ها در مسافت های بزرگ بسیار خاص و قابل توجه هستند. این سلول ها وظیفه انتقال داده‌های عصبی را بردوش دارند.
 آن‌ها این کار را از راه تولید و هدایت پالس الکتریکی مشخصی به نام پتانسیل عمل (اسپایک) انجام می‌دهند؛ اسپایک هایی در ولتاژ که می توانند در طول رشته های عصبی حرکت کنند. اسپایک ها گذار های سریع، و کوتاه مدت و نوک تیزی در ولتاژ الکتریکی (اسپایک های ولتاژ) و یا در جریان الکتریکی (اسپایک های جریان) هستند.
 \\

از اواخر قرن هجده نشان داده شده که حرکت پاهای قورباغه مرده با «الکتریسیته حیوان» ایجاد می شود، فرضیه های مختلفی درباره این که چگونه ممکن است اعصاب، شبیه سیم، پیام های الکتریکی را انتقال دهند و چگونه می توانند موجب انقباض عضلانی شوند، پیشنهاد شده است. از چند لحاظ، نورون بسیار شبیه دیگر انواع سلول ها است. آنچه نورون را مثلن از سلول کبد متمایز می کند، دو ویژگی کلیدی است:
\begin{itemize}
	\item{
غشا سلول تحریک پذیر است؛ یعنی ویژگی ای دارد که آن را برای ایجاد پیام های الکتریکی مبنای ارتباط ها توانمند می کند؛
	}
	\item {
نورون ها از نظر شکل می توانند خیلی متنوع و خیلی بلند باشند. 
	}
\end{itemize}
وقتی نورون ها را زیر میکروسکوپ نگاه می کنیم، مشخش می شود که آنها می توانند به اشکال متفاوت ظاهر شوند؛ ولی به هر حال تمام نورون ها الگوی بنیادینی مشابهی دارند و نواحی مختلف سلول (جسم سلولی یا سُما، اکسون، دندریت، پایانه ها وموارد دیگر)، کنش های ویژه ای را بر عهدا دارند. این آرایش در شکل نشان داده شده است.
\\
در اولین قدم سعی می کنیم که به کمک ریاضیات و فیزیک رفتار غشا نورون ها را مدل سازی کنیم.

\section{ غشا نورون}
غشا نورون از فسفولیپیدها تشکیل شده است. فسفولیپیدها مولکول های بزرگی هستند که دارای یک سر آب دوست و یک سر آب گریز می باشند. محیط درون و بیرون سلول ها همواره سراسر آب است به همین دلیل برای پایداری بیشتر، دو صفحه فسفولیپید به طوری قرار میگیرند که سر آبگریز آن ها به هم نزدیک و سر آب دوست آن ها در دو طرف قرار دارد. بنابراین غشا نورون تشکیل شده است از یک دو لایه فسفولیپید که سر آب دوست آن ها در دو طرف صفحه قرار دارد و سر آب گریز آن  ها دو وسط این صفحه قرار دارد.
\\
 غشا سلول قسمت درون سلول را از بیرون سلول جدا می کند. با اینکه در هر دو طرف غشا (دورن و بیرون)، ماده غالب آب است اما ترکیب آب درون و بیرون سلول با هم متفاوت است. ترکیب آب بیرون سلول بسیار شبیه به ترکیب آب دریاست، به این معنی که غلظت یون های سدیم و کلر در آن زیاد است؛ این مسئله را می توان به نحوی به خاطره سلول ها از تکامل نیز تعبیر کرد.  همانطور که می دانید، اولین سلول های زنده در آب دریا ها بوجود آمده اند و هنوز هم ترکیب آب بیرون سلول ها شبیه به محیط های اولیه ایست که سلول های اولیه در آن تشکیل شده اند. 
 \\
غشا سلول از چربی (فسفولیپید دولایه) تشکیل شده است که حالت عایق (نارسانا) را ایجاد می کند؛ مثل خازن که می تواند بار الکتریکی را ذخیره کند. گفتیم که غشا سلول علی الاصول نارساناست، اما غشای سلول عصبی پر از پرتئین هایی است که به شکل منافذ یا مجاری نفوذ، به یون ها اجازه عبور می دهند؛ این منافذ را کانال های یونی می نامند. گذر بار از کانال های یونی می تواند کنترل شود. این کنترل گذر بارها از کانال ها را می توان به وجود نوعی هوشمندی در سلول ها تعبیر کرد. 
توانایی ذخیره و اضافه شدن بار و سپس از دست دادن آن به آرامی با نفوذ و نشت از کانال ها اتفاق می افتد که برای کنشِ عصبی بسیار اساسی است. 
\subsection{کانال های پتاسیم}
علاوه بر نشت و نفوذ غیر انتخابی کانال ها، در غشا کانال هایی داریم که فقط به یون پتاسیم اجازه عبور می دهند. پمپ یونی غلظت پتاسیم را درون نورون ۳۰ تا ۴۰ بار بیشتر از بیرون نگه می دارد. 

\begin{equation}
 \Phi(r)=\frac{-GM}{\sqrt{r^{2}+b^{2}}}
 \label{eq:Plummer_pot}
\end{equation}
که توسط چگالی 
\ref{eq:Plummer_den}
ایجاد می شود
\begin{equation}
\rho(r)=\frac{3M}{4\pi b^{3}}\left( 1+\frac{r^{2}}{b^{2}} \right)^{-5/2}.
 \label{eq:Plummer_den}
\end{equation}
برای محاسبه مدار ها ابتدا باید به کمک مشتق گیری از پتانسیل، شتاب وارد بر یک ذره 
درون این پتانسیل را بدست آوریم.
\begin{equation}
 \vec{a} = - \nabla \Phi  \\
\label{eq:acceleration}
\end{equation}
\[
% f(x) =
\left\{
  \begin{array}{ll}
   \ddot{r} = -\frac{\partial}{\partial r} \Phi=\frac{-GMr}{(r^{2}+b^{2})^{3/2}}\\
   \ddot{\varphi}= 0 \Rightarrow \dot{\varphi}=cte. \Rightarrow \varphi=\alpha 
t+\varphi_{0}
  \end{array}
  \right.
\]
روابط
\ref{eq:acceleration}
، معادلات حرکت ذره درون پتانسیل 
\ref{eq:Plummer_pot}
هستند و به کمک آن ها می توانیم مدار ذره را در این پتانسیل به دست بیاوریم.

انتظار می رود که پتانسیل گرانشی مرتبط با کهکشان ها با زمان تغییر کند. تحت این شرایط کمیت 
هایی نظیر انرژی، نقاط اوج و حضیض و همچنین فرکانس تک تک مدار ها دیگر پایسته نیستند. 
بر این اساس مطالعه چگونگی تأثیرگذاری پتانسیل های وابسته به زمان روی ساختار مدارها از 
اهمیت خاصی برخوردار است. برای دستیابی به این هدف، مدلی ساده از یک کهکشان ماهواره ای را در 
نظر می گیریم که با یک کره پلامر یکپارچه می شود؛ و جرم کره پلامر بر اساس رابطه
\ref{eq:mass_varies}
تغییر می کند.
\begin{equation}
 \label{eq:mass_varies}
 M(t)=M_{0} \exp (t/t_{scale})
\end{equation}
\begin{latin}
	Hi I am saeed
\end{latin}

% \begin{thebibliography}{99}
% \begin{latin}
% 
% \end{latin}
% \end{thebibliography}



\end{document}